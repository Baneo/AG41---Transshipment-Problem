\documentclass[a4paper,10pt]{article}
%\documentclass[a4paper,10pt]{scrartcl}

\usepackage[utf8]{inputenc}
\sloppy
\title{AG41 - Problème de Transbordement}
\author{Guillaume PROST et Aurélien DURANCE}
\date{27 Mai 2016}

\pdfinfo{%
  /Title    ()
  /Author   ()
  /Creator  ()
  /Producer ()
  /Subject  ()
  /Keywords ()
}

\begin{document}
\maketitle
\section{Analyse Du Problème}

\hspace{0.5cm} Paramètres:\newline
F: nombre de fournisseurs\newline
D: nombre de destinataires\newline
P: nombre de plateformes\newline
$E_{ij}$: Edge allant de i vers j\newline
$g_{i}$: coût d'utilisation de la plateforme i\newline
$c_{ij}$: coût dfe transfert du node i vers le node j\newline
$h_{ij}$: coût unitaire du trajet du node i vers le node j par objet\newline
$t_{ij}$: temps de trajet du node i vers le node k en passant par le node j\newline
$s_{i}$: temps de transbordement\newline
$Cap_{D}$: capacité de reception d'un destinataire D\newline
$Cap_{F}$: capacité d'envoi d'un fournisseur F\newline
$u_{ij}$: capacité de l'edge ij\newline
T: Durée maximale acceptable de transbordement

Variables:\newline
$X_{ij}$: Nombre de paquets qui transitent de i vers j\newline
$Y_{ij}$: Variable binaire, vaut 1 si $X_{ij}$ est supérieur à 0, 0 sinon\newline
$Z_{i}$: Variable binaire, vaut 1 si la plateforme i est utilisée, 0 sinon

Fonction Objectif:\newline
La fonction objectif minimise les coûts de transfert des marchandises.

Contraintes:\newline
- Ne pas dépasser la capacité maximale par client\newline
- Ne pas dépasser la capacité maximale par fournisseurs\newline
- Ne pas dépasser la capacité maximale des edges\newline
- Respecter le temps d'execution
\newpage
\section{Modèle Mathématique}
\hspace{0.5cm}Fonction Objectif:\newline
min $z_{c} = \sum_{i=1}^{F} (\sum_{j=1}^{P} (\sum_{k=1}^{D} (X_{ij} \times h_{ij} + X_{jk} \times
h_{jk} + Y_{ij} \times c_{ij} + Y_{jk} \times c_{jk} + Y_{j} \times c_{j})))$

Contraintes:
\newline$ \sum_{i=1}^{i=F}(\sum_{j=1}^{j=P}(X_{jk})) = Cap_{D}$ pour tout k entier dans [1;D]
\newline$ \sum_{j=1}^{j=P}(\sum_{k=1}^{k=D}(X_{ij})) = Cap_{F}$ pour tout i entier dans [1;D]
\newline Pour chaque i,j: $X_{ij} <= u_{ij}$
\newline $ \sum_{i=1}^{F} (\sum_{j=1}^{P} (\sum_{k=1}^{D} (t_{ij} \times X_{ij} + s_{j} + t_{jk}
\times X_{jk}))) <= T$

\section{Algorithme de résolution}

\hspace{1cm}Afin de résoudre ce problème, nous avons préalablement choisi l'algorithme du Branch and Bound
qui nous permettait de trouver en théorie une solution optimale au bout d'un certain temps.
Nous avons décider dd'utiliser le language JAVA pour des raisons de facilité et de portabilité.\newline

\hspace{0.5cm}Dans un second temps, après avoir discuté un peu autour de nous, nous avons vu que beaucoup de 
personnes allaient utiliser cet algorithme, nous avons donc décidé de changer de méthode afin de 
fournir un travail original et un challenge plus important (étant donné que le Branch and Bound 
a une efficacité prouvée, nous voulions voir si un autre algorithme aurait pu concurrencer le 
Branch and Bound).\newline

\hspace{0.5cm}Nous avons remarqué que ce problème de transbordement pouvait se rapporter à un problème de flux
maximal à coût minimal (étant donné qu'il faut faire transiter une ressource d'un point A vers
un point B) donc nous nous sommes concentrés sur l'implémentation de cet algorithme. Nous avons
donc commencé par reconstruire un algorithme de lecture des fichiers de données, qui nous a permis 
d'implémenter le programme de la manière qui nous semblait la plus simple en terme de structures 
(objets et méthodes), puis nous avons crée la solution initiale et enfin nous avons 
implémenté notre algorithme afin de répondre aux problèmes proposés.
\section{Tests et résultats}

Voici le tableau récapitulatif de nos tests avec la version stable de notre programme, on y 
retrouve le nom du fichier utilisé, le nombre de nodes et d'edges, le temps d'exécution, la 
solution trouvée et le nombre de paquets restants (dans le cas d'un algorithme avec un problème de
priorité).\newline

\begin{tabular}{|c|c|c|c|c|c|}
  \hline
  Nom du Modèle & Nb Nodes & Nb Edges & Tps exécution & Solution & Paquets restants\\
  \hline
  tshp10-01 & 10 & 9 & 0.0 & 3938 & 0\\
  tshp010-01 & 10 & 21 & 0.0 & 4856 & 0\\
  tshp10-02 & 10 & 16 & 0.0 & 2990 & 0\\
  tshp010-02 & 10 & 21 & 0.0 & 2101 & 0\\
  tshp010-03 & 10 & 16 & 0.0 & 1949 & 0\\
  tshp010-04 & 10 & 16 & 0.0 & 2110 & 0\\
  tshp010-05 & 10 & 16 & 0.0 & 2864 & 0\\
  tshp020-01 & 20 & 75 & 0.0 & 4106 & 0\\
  tshp020-02 & 20 & 91 & 0.0 & 4902 & 0\\
  tshp020-03 & 20 & 75 & 0.0 & 5163 & 0\\
  tshp020-04 & 20 & 75 & 0.0 & 3854 & 0\\
  tshp020-05 & 20 & 36 & 0.0 & 11120 & 0\\
  tshp050-01 & 50 & 576 & 0.0 & 13671 & 18\\
  tshp050-02 & 50 & 504 & 0.0 & 15147 & 31\\
  tshp050-03 & 50 & 301 & 0.0 & 9836 & 0\\
  tshp050-04 & 50 & 525 & 0.0 & 13533 & 0\\
  tshp050-05 & 50 & 301 & 0.0 & 12956 & 0\\
  tshp100-01 & 100 & 979 & 6.0 & 24992 & 25\\
  tshp100-02 & 100 & 1275 & 7.0 & 30459 & 5\\
  tshp100-03 & 100 & 1476 & 6.0 & 28657 & 107\\
  tshp100-04 & 100 & 2275 & 9.0 & 19066 & 119\\
  tshp100-05 & 100 & 2275 & 9.0 & 23289 & 102\\
  \hline
\end{tabular}

\section{Avantages, Inconvénients et Problèmes}

\hspace{1cm}Concernant les inconvénients de cet algorithme, on peut noter que c'est un
algorithme de construction, donc la contrainte de temps d'exécution du 
programme est un problème majeur étant donné que si on interrompt le programme avant qu'il
ait fini sa construction, on a pas de solution. Deuxième problème à noter, 
le temps d'exécution en lui même. En effet, plus il y aura de noeuds dans le problème, plus 
l'algorithme va prendre de temps pour résoudre le problème (comme on peut le voir dans le tableau
ci-dessus, on a des temps qui vont jusqu'à 9 secondes pour un modèle à 100 noeuds, et un modèle à 500 noeuds dure plus de 3h). 

\hspace{0.5cm}Puis, concernant les avantages, notre algorithme est très efficace en terme de
temps quand il s'agit de traiter un problème avec peu de noeuds (comme on peut le voir dans 
le tableau ci-dessus, il faut aller à un problème à 100 noeuds pour obtenir un temps d'exécution 
notable). Ensuite, comme c'est un algorithme de construction, il trouve directement
la meilleure solution optimale, donc la qualité de la solution est meilleure qu'avec d'autres
algorithmes. C'est également un algorithme qui va toujours trouver la solution optimale peut importe le 
nombre de noeuds.\newline

\hspace{0.5cm}Et enfin, nous allons parler des problèmes que nous avons rencontré ainsi que des
solutions que nous avons mises en place pour y palier. \\ Concernant le problème de la contrainte de 
temps qui induit de ne pas trouver de solution si le temps est trop court, nous avons décidé de 
mettre en place une solution initiale (comme dans les algorithmes d'optimisation ne reposant pas sur 
une construction) afin d'avoir une solution quel que soit le temps d'execution du programme. Dans 
ce cas, la solution est considérée comme binaire: soit on obtient une solution initiale par défaut, qui n'est pas 
optimisée (mais qui est une solution tout de même), soit on obtient une solution optimale. 
Un autre problème que nous avons rencontré est le problème de la construction en elle même.
En effet, l'algorithme parcourt tous les chemins viables et choisit le plus optimal pour faire
passer des paquets. Mais il ne prend pas en compte le fait qu'un chemin peut ne mener qu'à un seul 
client, et si ce client est déjà satisfait par un autre fournisseur, ce chemin sera bloqué et 
des paquets pourront ne pas être transférés. \\Afin de contrer ce problème, nous avons essayé de 
détecter ces cas de figure afin de remplir d'abord les chemins uniques, et ensuite les chemins 
optimaux, afin que tous les paquets puissent partir. Nous avons remarqué que ce problème de blocage
était présent dans les problèmes à 50 noeuds et plus (correspondant dans le tableau aux lignes ou il
y a des paquets restants). Etant donné que les chemins sont parcourus 
les uns après les autres, il est logique qu'il y ait des problèmes. Pour améliorer notre programme,
nous avons donc pensé a implémenter une chaîne améliorante, adaptée au projet, qui prendrait en compte les chemins uniques et 
en qui implémenterai un système de priorités sur ces chemins. Malheureusement face à la complexité du 
problème, nous avons été dans l'incapacité de mener à bout cette solution.


\end{document}
